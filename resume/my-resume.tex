%%%%%%%%%%%%%%%%%%%%%%%%%%%%%%%%%%%%%%%%%
% Medium Length Professional CV
% LaTeX Template
% Version 2.0 (8/5/13)
%
% This template has been downloaded from:
% http://www.LaTeXTemplates.com
%
% Original author:
% Trey Hunner (http://www.treyhunner.com/)
%
% Important note:
% This template requires the resume.cls file to be in the same directory as the
% .tex file. The resume.cls file provides the resume style used for structuring the
% document.
%
%%%%%%%%%%%%%%%%%%%%%%%%%%%%%%%%%%%%%%%%%

%----------------------------------------------------------------------------------------
%	PACKAGES AND OTHER DOCUMENT CONFIGURATIONS
%----------------------------------------------------------------------------------------

\documentclass{resume} % Use the custom resume.cls style
\usepackage{hyperref}
\usepackage{fontawesome}

\usepackage[left=0.35in,top=0.3in,right=0.35in,bottom=0.1in]{geometry} % Document margins

\name{Qiang Zhang} % Your name
\address{3333 North Charles Street, Baltimore, MD 21218 \\ (667) 212-6099 \\ vegito2002@gmail.com  \href{https://github.com/vegito2002}{\textcolor{blue}{\faGithub} \href{https://stackoverflow.com/users/7088096/qiang-zhang}{\textcolor{blue}{\faStackOverflow}}} \href{https://www.linkedin.com/in/qiang-zhang-855615117/}{\textcolor{blue}{\faLinkedinSquare}}} % Your address
% \address{(408) 769-8746 \\ yifan.ge@outlook.com} % Your phone number and email

\begin{document}

%----------------------------------------------------------------------------------------
%	EDUCATION SECTION
%----------------------------------------------------------------------------------------

\begin{rSection}{Education}
{\bf Johns Hopkins University} \hfill {\bf Baltimore, MD} \\
{Master of Science in Engineering (M.S.E.) in Computer Science} {\em GPA: 3.74/4.00} \hfill {Sept. 2016 - May 2018}

{\bf Southeast University} \hfill {\bf Nanjing, China} \\
{B.E. in Measuring Control Technology \& Instruments} {\em GPA: 3.1/4.0} \hfill {Aug. 2007 - Jun. 2011}
\end{rSection}


%----------------------------------------------------------------------------------------
%	WORK EXPERIENCE SECTION
%----------------------------------------------------------------------------------------

\begin{rSection}{Experience}

\begin{rSubsection}{Nanjing Institute of Measurement And Testing Technology}{2012 - 2014}{Calibrating Engineer}{Nanjing, China}
\item Legally binding calibration and certification for over 8000 pieces of academic or industrial instruments.
\item Automated workflow for the department resulting in significant productivity gain, and won Anual Best 10 Employee (out of 200+) of year 2013, an unprecedented achievement for a second year.
\end{rSubsection}

\begin{rSubsection}{Lab of Dr. Xu Li of Southeast University}{2012 - 2013}{Part-time Research Assistant}{Nanjing, China}
\item Participation in project: Truck \& Train Motile Property Testing System.
% \item Technical interviewer for Software Engineering roles.
\end{rSubsection}
%------------------------------------------------
\end{rSection}

%----------------------------------------------------------------------------------------
%	PUBLICATIONS
%----------------------------------------------------------------------------------------

% \begin{rSection}{Publications and Presentations}
% {\bf Oral Presentation:} Xi Zhou; {\bf Yifan Ge}; Xuxu Chen; Yinan Jing; Weiwei Sun; , {\em `A Distributed Cache Based Reliable Service Execution and Recovery Approach in MANETs'}, The 6th IEEE International Conference on Asia-Pacific Services Computing Conference, Jeju-do, South Korea, Dec. 2011.

% {\bf Poster Presentation:} Xi Zhou; {\bf Yifan Ge}; Xuxu Chen; Yinan Jing; Weiwei Sun; , {\em `SMF: A Novel Lightweight Reliable Service Discovery Approach in MANET'}, The 7th International Conference on Wireless Communications, Networking and Mobile Computing, Wuhan, China, Sept. 2011.
% \end{rSection}

%----------------------------------------------------------------------------------------
%	SELECTED PROJECTS
%----------------------------------------------------------------------------------------
\begin{rSection}{Project Highlights}

\begin{rSubsection}{Survival Maps \href{https://github.com/cc941201/SurvivalMaps}{\textcolor{blue}{\faGithub}}}{Team: Guoye Zhang, {\bf Qiang Zhang}, Neha Kulkarni, Channing Kimble-Brown, Jeana Yee}{Bootstrap JavaSpark TravisCI Maven Heroku iOS sql2o JUnit SQLite MVC MapQuest RESTful}{2016, Baltimore, MD}
\item Large-scale course project of OOSE where we, out of real-life concern built a navigation app that cares about security no less than speed, during a fully fleshed 6-iteration development: requirement analysis, design, implement, automated testing and deployment.
\item Innovative features: crime heatmap, routing security preference, sensitivity to crime threat of different time of a day etc.
\item Solely responsible for UML design; significant participation in back-end server design and implementation; primarily responsible for backend data processing and data source updating component.
\item Designed a recursive algorithm to integrate inconsistent traffic data and crime data from different sources.
\end{rSubsection}
%--------------------------------
\begin{rSubsection}{Multi Diff \href{https://github.com/vegito2002/multi-diff}{\textcolor{blue}{\faGithub}}}{Team: Guoye Zhang, {\bf Qiang Zhang}}{diff Algorithm HTML MinHash Maven ZIMPL}{2017, Baltimore, MD}
\item We re-assess the legacy utility \texttt{diff} and augment it with additional feature and algorithmic enhancement.
\item Developed new algorithm \texttt{BetterDiff} that intelligently retain bracket pairing while calculating edit distance.
\item Enhanced the system to be able to track editting ancestry in large multi-file hierarchially structured file system, producing standard \texttt{diff} patch files or intuitive side-by-side visualization.
\end{rSubsection}
% %--------------------------------
\begin{rSubsection}{HMM EM Tagger \href{https://github.com/vegito2002/hmm-em-tagger}{\textcolor{blue}{\faGithub}}}{Myself}{NLP Tagger Supervised-Learning Semi-Supervised-Learning HMM EM}{2017, Baltimore, MD}
\item Compact implementation of a Hidden Markov model based tagger that can do not only supervised learning with viterbi decoding or posterior decoding, but also semi-supervised learning with expectation–maximization algorithm.
\item Innovative optimization of word-similarity based tag dictionary pruning that speeds the tagger up by 50\% - 80\% depending on task.
\end{rSubsection}
% %--------------------------------
\begin{rSubsection}{Padding Oracle Attack Demo \href{https://github.com/vegito2002/padding-oracle-attack-demo}{\textcolor{blue}{\faGithub}}}{Myself}{Cryptography Security HMAC SHA AES-CBC}{2017, Baltimore, MD}
\item Down from scratch Golang implementation of a typical padding oracle attack process. Includes both an authenticated encryption component and the adversary component.
\end{rSubsection}
% %--------------------------------
\end{rSection}
%----------------------------------------------------------------------------------------
%	TECHNICAL STRENGTHS SECTION
%----------------------------------------------------------------------------------------

\begin{rSection}{Technical Skills (in order of nonascending proficiency)}

\begin{tabular}{ @{} >{\bfseries}l @{\hspace{3ex}} l }
Languages & Java, Go, C, Python, SQL, Ocaml, Swift, JavaScript, jQuery \\
Frameworks \& Platforms & JavaSpark, Flask \\
Tools & Git, Sublime Text, \textsc{Latex}, Linux, shell, Microsoft Office, Intellij IDEA, MySQL
\end{tabular}

\end{rSection}

%----------------------------------------------------------------------------------------

\begin{rSection}{Coursework Highlights (Graduate Level)}

\begin{tabular}{l l l}
Algorithms & Object Oriented Software Engineering (OOSE) & Databases \\
Computer Networks & Principles of Programming Language & Declarative Methods \\
Natural Language Processing & Practical Cryptographic Systems & Operating Systems & Machine Learning\\
\end{tabular}
\end{rSection}

% %----------------------------------------------------------------------------------------
% \begin{rSection}{Miscellaneous}
% Native Chinese speaker

% Section content\ldots 

% \end{rSection}

%----------------------------------------------------------------------------------------
%	
%----------------------------------------------------------------------------------------

%\begin{rSection}{Section Name}

%Section content\ldots

%\end{rSection}

%----------------------------------------------------------------------------------------

\end{document}
